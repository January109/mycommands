% IMPORTANT NOTE
% --------------
% For your .tex files to be able to find this, it either needs to
% be in the same directory, or located as per
% https://tex.stackexchange.com/a/1167/243466 .
% You can make a symlink (or equivalent for non-Linux) so that the file
% can stay elsewhere
% Use with % IMPORTANT NOTE
% --------------
% For your .tex files to be able to find this, it either needs to
% be in the same directory, or located as per
% https://tex.stackexchange.com/a/1167/243466 .
% You can make a symlink (or equivalent for non-Linux) so that the file
% can stay elsewhere
% Use with % IMPORTANT NOTE
% --------------
% For your .tex files to be able to find this, it either needs to
% be in the same directory, or located as per
% https://tex.stackexchange.com/a/1167/243466 .
% You can make a symlink (or equivalent for non-Linux) so that the file
% can stay elsewhere
% Use with % IMPORTANT NOTE
% --------------
% For your .tex files to be able to find this, it either needs to
% be in the same directory, or located as per
% https://tex.stackexchange.com/a/1167/243466 .
% You can make a symlink (or equivalent for non-Linux) so that the file
% can stay elsewhere
% Use with \input{commands.tex}

\usepackage{amssymb, amsmath, amsthm} % https://tex.stackexchange.com/a/32102
\usepackage{mathtools} % makes maths nicer
\usepackage[margin=1in]{geometry} % Gives smaller margins
\usepackage[colorlinks=true,linkcolor=blue,citecolor=magenta]{hyperref} % For urls and to make references nicer

% http://www.ams.org/arc/tex/amscls/amsthdoc.pdf for theorems

% Possible packages:
% \usepackage{biblatex} % For bibliographies
\usepackage{graphicx} % For including images
\usepackage{float} % For [H] option for figures
\usepackage{enumitem} % For controlling lists better: see environments defiend below
\usepackage{xparse} % For helping with arguments of new environments/commands
\usepackage{bm} % For better bold maths (use \bm)
\usepackage{caption} % Makes sure that empty figure captions don't have a ":"
\usepackage{wrapfig} % to make wrapfigures - pretty self-explanatory
\usepackage{pdfpages} % To include other pdf pages (e.g. coversheets), use \includepdf[pages=-]{file.pdf} to include all of file.pdf .
\usepackage{fancyvrb} % To do fancy stuff with verbatim environments
\usepackage{polynom} % For poly long div with \polylongdiv

% Cleveref package for nice references
\usepackage[capitalise,noabbrev]{cleveref} % capitalise figure names and don't abbreviate
\crefname{equation}{}{} % Remove 'equation' from equation references

% Tikz library for drawings
\usepackage{tikz}
\usetikzlibrary{arrows.meta} % For drawing arrows in tikz
\usetikzlibrary{calc} % For calculating coordinates in tikz

\usepackage{lmodern} % Different monospace font
\usepackage[T1]{fontenc} % idk why
\usepackage{listings} % For including code
\usepackage{lstautogobble} % Remove leading tabs
\lstset{basicstyle=\ttfamily\bfseries, columns=fixed, tabsize=4, autogobble=true}
% You might want to add an autocomplete to your editor for:
% \begin{center}\vspace{-1.5em}\begin{tabular}{c}\begin{lstlisting}
% \end{lstlisting}\end{tabular}\vspace{-1em}\end{center}
% That will make a centered code listing.

% Change the document-wide spacings
\setlength{\parindent}{0in} % Make sure paragraphs don't indent
\setlength{\parskip}{1em} % Put a space between paragraphs
% Ensure that display maths environments aren't too spaced out
\setlength{\abovedisplayskip}{0.8em}
\setlength{\abovedisplayshortskip}{0.5em}
\setlength{\belowdisplayskip}{0.8em}
\setlength{\belowdisplayshortskip}{0.5em}

\allowdisplaybreaks % Lets equations run over pages

% amsthm definitions
\newtheoremstyle{plain}
	{2\topsep}  % ABOVESPACE
	{0}   		% BELOWSPACE
	{\itshape}  % BODYFONT
	{0pt}       % INDENT (empty value is the same as 0pt)
	{\bfseries} % HEADFONT
	{.}         % HEADPUNCT
	{5pt plus 1pt minus 1pt} % HEADSPACE
	{}          % CUSTOM-HEAD-SPEC

\newtheoremstyle{case}
	{6pt}
	{6pt}
	{}
	{}
	{\bfseries}
	{:}
	{.5em}
	{}

\theoremstyle{plain}
\newtheorem{thm}{Theorem}
\newtheorem{lem}{Lemma}
\newtheorem{prop}{Proposition}

\theoremstyle{definition}
\newtheorem{defn}{Definition}

\theoremstyle{case}
\newtheorem{case}{Case}
\newtheorem*{BC}{Base Case}
\newtheorem*{BCs}{Base Cases}
\newtheorem*{IS}{Inductive Step}
\newtheorem*{IF}{Proof of ``if''}
\newtheorem*{OI}{Proof of ``only if''}

% For making an enumerate environment with the list items set into the margins of the page.
\newenvironment{insetenum}[1][]
{\begin{enumerate}[leftmargin=0pt,#1]}
{\end{enumerate}}

% For indenting things
\newenvironment{indentblock}
{\begin{itemize}[parsep=\parskip,topsep=0pt,partopsep=0pt,leftmargin=0.5cm]\item[]}
{\end{itemize}}

% Defines a compact enumerate and itemize with an optional argument to specify the top-spacing (default is -1em to cancel the \parskip).
\newenvironment{compenum}[1][-0.5em]
{\begin{enumerate}[noitemsep,topsep=#1,parsep=0.5ex,partopsep=0pt]}
{\end{enumerate}}

\newenvironment{compitem}[1][-0.5em]
{\begin{itemize}[noitemsep,topsep=#1,parsep=0pt,partopsep=0pt]}
{\end{itemize}}

% Use this when there are multiple steps in a proof of something, like each
% side of an inclusion or proving an isomorphism by proving homomorphism,
% surjective, injective.
\newenvironment{pflist}[1][-0.5em]
{\begin{itemize}[topsep=#1,partopsep=0pt]}
{\end{itemize}}

% https://tex.stackexchange.com/a/35934/243466
% I don't really use this much
\usepackage{changepage}
\newenvironment{ind}[1][2em]{\begin{adjustwidth}{#1}{}}{\end{adjustwidth}}

% Complexity classes
\renewcommand{\P}[0]{\ensuremath{\mathcal{P}}}
\newcommand{\NP}[0]{\ensuremath{\mathcal{NP}}}

% Defining maths symbol shortcuts
\newcommand{\N}{\mathbb{N}}
\newcommand{\Z}{\mathbb{Z}}
\newcommand{\R}{\mathbb{R}}
\newcommand{\F}{\mathbb{F}}
\newcommand{\C}{\mathbb{C}}
\newcommand{\Q}{\mathbb{Q}}
\newcommand{\e}{\varepsilon}
\newcommand{\p}{\varphi}
\newcommand{\bfb}{\mathbf{b}}
\NewDocumentCommand{\ddx}{s m}{
	\IfBooleanTF {#1}
		{ \frac{d}{dx}#2 }
		{ \frac{d}{dx}\left(#2\right) }
}
\newcommand{\imp}{\Longrightarrow}
\newcommand{\stline}{\;\middle\vert\;}
\newcommand{\floor}[1]{\left\lfloor#1\right\rfloor}
\newcommand{\ceil}[1]{\left\lceil#1\right\rceil}
\newcommand{\inv}{^{-1}}
\newcommand{\norm}[1]{\left\lVert#1\right\rVert}
\newcommand{\abs}[1]{\left\lvert#1\right\rvert}
\newcommand{\lrangle}[1]{\left\langle#1\right\rangle} % Inner product
\newcommand{\sub}{\leqslant}
\newcommand{\nsub}{\triangleleft}
\newcommand{\card}[1]{\overline{\overline{#1}}}

% Defining some maths operators
\DeclareMathOperator{\tr}{tr}
\DeclareMathOperator{\lcm}{lcm} % gcd already exists
\DeclareMathOperator{\proj}{proj}
\DeclareMathOperator{\sgn}{sgn}
\DeclareMathOperator{\Span}{Span\,}
\DeclareMathOperator{\Nul}{Nul}
\DeclareMathOperator{\Col}{Col}
\DeclareMathOperator{\Row}{Row}
\DeclareMathOperator{\Area}{Area}
\DeclareMathOperator{\Arg}{Arg}
\DeclareMathOperator{\Img}{Im} % \Im is already something
\DeclareMathOperator{\Ker}{Ker}
\DeclareMathOperator{\Sym}{Sym}
\DeclareMathOperator{\Stab}{Stab}
\DeclareMathOperator{\Null}{Null}
\DeclareMathOperator{\Range}{Range}
\DeclareMathOperator{\Ann}{Ann}

% Define a question environment
\NewDocumentEnvironment{question}{m o}
{\IfNoValueTF{#2}{\section*{Question #1}}{\section*{Question #1: #2}}}
{
	\clearpage\newpage
	\setcounter{thm}{0}
	\setcounter{lem}{0}
	\setcounter{prop}{0}
	\setcounter{defn}{0}
	\setcounter{case}{0}
	\setcounter{equation}{0}
}


\usepackage{amssymb, amsmath, amsthm} % https://tex.stackexchange.com/a/32102
\usepackage{mathtools} % makes maths nicer
\usepackage[margin=1in]{geometry} % Gives smaller margins
\usepackage[colorlinks=true,linkcolor=blue,citecolor=magenta]{hyperref} % For urls and to make references nicer

% http://www.ams.org/arc/tex/amscls/amsthdoc.pdf for theorems

% Possible packages:
% \usepackage{biblatex} % For bibliographies
\usepackage{graphicx} % For including images
\usepackage{float} % For [H] option for figures
\usepackage{enumitem} % For controlling lists better: see environments defiend below
\usepackage{xparse} % For helping with arguments of new environments/commands
\usepackage{bm} % For better bold maths (use \bm)
\usepackage{caption} % Makes sure that empty figure captions don't have a ":"
\usepackage{wrapfig} % to make wrapfigures - pretty self-explanatory
\usepackage{pdfpages} % To include other pdf pages (e.g. coversheets), use \includepdf[pages=-]{file.pdf} to include all of file.pdf .
\usepackage{fancyvrb} % To do fancy stuff with verbatim environments
\usepackage{polynom} % For poly long div with \polylongdiv

% Cleveref package for nice references
\usepackage[capitalise,noabbrev]{cleveref} % capitalise figure names and don't abbreviate
\crefname{equation}{}{} % Remove 'equation' from equation references

% Tikz library for drawings
\usepackage{tikz}
\usetikzlibrary{arrows.meta} % For drawing arrows in tikz
\usetikzlibrary{calc} % For calculating coordinates in tikz

\usepackage{lmodern} % Different monospace font
\usepackage[T1]{fontenc} % idk why
\usepackage{listings} % For including code
\usepackage{lstautogobble} % Remove leading tabs
\lstset{basicstyle=\ttfamily\bfseries, columns=fixed, tabsize=4, autogobble=true}
% You might want to add an autocomplete to your editor for:
% \begin{center}\vspace{-1.5em}\begin{tabular}{c}\begin{lstlisting}
% \end{lstlisting}\end{tabular}\vspace{-1em}\end{center}
% That will make a centered code listing.

% Change the document-wide spacings
\setlength{\parindent}{0in} % Make sure paragraphs don't indent
\setlength{\parskip}{1em} % Put a space between paragraphs
% Ensure that display maths environments aren't too spaced out
\setlength{\abovedisplayskip}{0.8em}
\setlength{\abovedisplayshortskip}{0.5em}
\setlength{\belowdisplayskip}{0.8em}
\setlength{\belowdisplayshortskip}{0.5em}

\allowdisplaybreaks % Lets equations run over pages

% amsthm definitions
\newtheoremstyle{plain}
	{2\topsep}  % ABOVESPACE
	{0}   		% BELOWSPACE
	{\itshape}  % BODYFONT
	{0pt}       % INDENT (empty value is the same as 0pt)
	{\bfseries} % HEADFONT
	{.}         % HEADPUNCT
	{5pt plus 1pt minus 1pt} % HEADSPACE
	{}          % CUSTOM-HEAD-SPEC

\newtheoremstyle{case}
	{6pt}
	{6pt}
	{}
	{}
	{\bfseries}
	{:}
	{.5em}
	{}

\theoremstyle{plain}
\newtheorem{thm}{Theorem}
\newtheorem{lem}{Lemma}
\newtheorem{prop}{Proposition}

\theoremstyle{definition}
\newtheorem{defn}{Definition}

\theoremstyle{case}
\newtheorem{case}{Case}
\newtheorem*{BC}{Base Case}
\newtheorem*{BCs}{Base Cases}
\newtheorem*{IS}{Inductive Step}
\newtheorem*{IF}{Proof of ``if''}
\newtheorem*{OI}{Proof of ``only if''}

% For making an enumerate environment with the list items set into the margins of the page.
\newenvironment{insetenum}[1][]
{\begin{enumerate}[leftmargin=0pt,#1]}
{\end{enumerate}}

% For indenting things
\newenvironment{indentblock}
{\begin{itemize}[parsep=\parskip,topsep=0pt,partopsep=0pt,leftmargin=0.5cm]\item[]}
{\end{itemize}}

% Defines a compact enumerate and itemize with an optional argument to specify the top-spacing (default is -1em to cancel the \parskip).
\newenvironment{compenum}[1][-0.5em]
{\begin{enumerate}[noitemsep,topsep=#1,parsep=0.5ex,partopsep=0pt]}
{\end{enumerate}}

\newenvironment{compitem}[1][-0.5em]
{\begin{itemize}[noitemsep,topsep=#1,parsep=0pt,partopsep=0pt]}
{\end{itemize}}

% Use this when there are multiple steps in a proof of something, like each
% side of an inclusion or proving an isomorphism by proving homomorphism,
% surjective, injective.
\newenvironment{pflist}[1][-0.5em]
{\begin{itemize}[topsep=#1,partopsep=0pt]}
{\end{itemize}}

% https://tex.stackexchange.com/a/35934/243466
% I don't really use this much
\usepackage{changepage}
\newenvironment{ind}[1][2em]{\begin{adjustwidth}{#1}{}}{\end{adjustwidth}}

% Complexity classes
\renewcommand{\P}[0]{\ensuremath{\mathcal{P}}}
\newcommand{\NP}[0]{\ensuremath{\mathcal{NP}}}

% Defining maths symbol shortcuts
\newcommand{\N}{\mathbb{N}}
\newcommand{\Z}{\mathbb{Z}}
\newcommand{\R}{\mathbb{R}}
\newcommand{\F}{\mathbb{F}}
\newcommand{\C}{\mathbb{C}}
\newcommand{\Q}{\mathbb{Q}}
\newcommand{\e}{\varepsilon}
\newcommand{\p}{\varphi}
\newcommand{\bfb}{\mathbf{b}}
\NewDocumentCommand{\ddx}{s m}{
	\IfBooleanTF {#1}
		{ \frac{d}{dx}#2 }
		{ \frac{d}{dx}\left(#2\right) }
}
\newcommand{\imp}{\Longrightarrow}
\newcommand{\stline}{\;\middle\vert\;}
\newcommand{\floor}[1]{\left\lfloor#1\right\rfloor}
\newcommand{\ceil}[1]{\left\lceil#1\right\rceil}
\newcommand{\inv}{^{-1}}
\newcommand{\norm}[1]{\left\lVert#1\right\rVert}
\newcommand{\abs}[1]{\left\lvert#1\right\rvert}
\newcommand{\lrangle}[1]{\left\langle#1\right\rangle} % Inner product
\newcommand{\sub}{\leqslant}
\newcommand{\nsub}{\triangleleft}
\newcommand{\card}[1]{\overline{\overline{#1}}}

% Defining some maths operators
\DeclareMathOperator{\tr}{tr}
\DeclareMathOperator{\lcm}{lcm} % gcd already exists
\DeclareMathOperator{\proj}{proj}
\DeclareMathOperator{\sgn}{sgn}
\DeclareMathOperator{\Span}{Span\,}
\DeclareMathOperator{\Nul}{Nul}
\DeclareMathOperator{\Col}{Col}
\DeclareMathOperator{\Row}{Row}
\DeclareMathOperator{\Area}{Area}
\DeclareMathOperator{\Arg}{Arg}
\DeclareMathOperator{\Img}{Im} % \Im is already something
\DeclareMathOperator{\Ker}{Ker}
\DeclareMathOperator{\Sym}{Sym}
\DeclareMathOperator{\Stab}{Stab}
\DeclareMathOperator{\Null}{Null}
\DeclareMathOperator{\Range}{Range}
\DeclareMathOperator{\Ann}{Ann}

% Define a question environment
\NewDocumentEnvironment{question}{m o}
{\IfNoValueTF{#2}{\section*{Question #1}}{\section*{Question #1: #2}}}
{
	\clearpage\newpage
	\setcounter{thm}{0}
	\setcounter{lem}{0}
	\setcounter{prop}{0}
	\setcounter{defn}{0}
	\setcounter{case}{0}
	\setcounter{equation}{0}
}


\usepackage{amssymb, amsmath, amsthm} % https://tex.stackexchange.com/a/32102
\usepackage{mathtools} % makes maths nicer
\usepackage[margin=1in]{geometry} % Gives smaller margins
\usepackage[colorlinks=true,linkcolor=blue,citecolor=magenta]{hyperref} % For urls and to make references nicer

% http://www.ams.org/arc/tex/amscls/amsthdoc.pdf for theorems

% Possible packages:
% \usepackage{biblatex} % For bibliographies
\usepackage{graphicx} % For including images
\usepackage{float} % For [H] option for figures
\usepackage{enumitem} % For controlling lists better: see environments defiend below
\usepackage{xparse} % For helping with arguments of new environments/commands
\usepackage{bm} % For better bold maths (use \bm)
\usepackage{caption} % Makes sure that empty figure captions don't have a ":"
\usepackage{wrapfig} % to make wrapfigures - pretty self-explanatory
\usepackage{pdfpages} % To include other pdf pages (e.g. coversheets), use \includepdf[pages=-]{file.pdf} to include all of file.pdf .
\usepackage{fancyvrb} % To do fancy stuff with verbatim environments
\usepackage{polynom} % For poly long div with \polylongdiv

% Cleveref package for nice references
\usepackage[capitalise,noabbrev]{cleveref} % capitalise figure names and don't abbreviate
\crefname{equation}{}{} % Remove 'equation' from equation references

% Tikz library for drawings
\usepackage{tikz}
\usetikzlibrary{arrows.meta} % For drawing arrows in tikz
\usetikzlibrary{calc} % For calculating coordinates in tikz

\usepackage{lmodern} % Different monospace font
\usepackage[T1]{fontenc} % idk why
\usepackage{listings} % For including code
\usepackage{lstautogobble} % Remove leading tabs
\lstset{basicstyle=\ttfamily\bfseries, columns=fixed, tabsize=4, autogobble=true}
% You might want to add an autocomplete to your editor for:
% \begin{center}\vspace{-1.5em}\begin{tabular}{c}\begin{lstlisting}
% \end{lstlisting}\end{tabular}\vspace{-1em}\end{center}
% That will make a centered code listing.

% Change the document-wide spacings
\setlength{\parindent}{0in} % Make sure paragraphs don't indent
\setlength{\parskip}{1em} % Put a space between paragraphs
% Ensure that display maths environments aren't too spaced out
\setlength{\abovedisplayskip}{0.8em}
\setlength{\abovedisplayshortskip}{0.5em}
\setlength{\belowdisplayskip}{0.8em}
\setlength{\belowdisplayshortskip}{0.5em}

\allowdisplaybreaks % Lets equations run over pages

% amsthm definitions
\newtheoremstyle{plain}
	{2\topsep}  % ABOVESPACE
	{0}   		% BELOWSPACE
	{\itshape}  % BODYFONT
	{0pt}       % INDENT (empty value is the same as 0pt)
	{\bfseries} % HEADFONT
	{.}         % HEADPUNCT
	{5pt plus 1pt minus 1pt} % HEADSPACE
	{}          % CUSTOM-HEAD-SPEC

\newtheoremstyle{case}
	{6pt}
	{6pt}
	{}
	{}
	{\bfseries}
	{:}
	{.5em}
	{}

\theoremstyle{plain}
\newtheorem{thm}{Theorem}
\newtheorem{lem}{Lemma}
\newtheorem{prop}{Proposition}

\theoremstyle{definition}
\newtheorem{defn}{Definition}

\theoremstyle{case}
\newtheorem{case}{Case}
\newtheorem*{BC}{Base Case}
\newtheorem*{BCs}{Base Cases}
\newtheorem*{IS}{Inductive Step}
\newtheorem*{IF}{Proof of ``if''}
\newtheorem*{OI}{Proof of ``only if''}

% For making an enumerate environment with the list items set into the margins of the page.
\newenvironment{insetenum}[1][]
{\begin{enumerate}[leftmargin=0pt,#1]}
{\end{enumerate}}

% For indenting things
\newenvironment{indentblock}
{\begin{itemize}[parsep=\parskip,topsep=0pt,partopsep=0pt,leftmargin=0.5cm]\item[]}
{\end{itemize}}

% Defines a compact enumerate and itemize with an optional argument to specify the top-spacing (default is -1em to cancel the \parskip).
\newenvironment{compenum}[1][-0.5em]
{\begin{enumerate}[noitemsep,topsep=#1,parsep=0.5ex,partopsep=0pt]}
{\end{enumerate}}

\newenvironment{compitem}[1][-0.5em]
{\begin{itemize}[noitemsep,topsep=#1,parsep=0pt,partopsep=0pt]}
{\end{itemize}}

% Use this when there are multiple steps in a proof of something, like each
% side of an inclusion or proving an isomorphism by proving homomorphism,
% surjective, injective.
\newenvironment{pflist}[1][-0.5em]
{\begin{itemize}[topsep=#1,partopsep=0pt]}
{\end{itemize}}

% https://tex.stackexchange.com/a/35934/243466
% I don't really use this much
\usepackage{changepage}
\newenvironment{ind}[1][2em]{\begin{adjustwidth}{#1}{}}{\end{adjustwidth}}

% Complexity classes
\renewcommand{\P}[0]{\ensuremath{\mathcal{P}}}
\newcommand{\NP}[0]{\ensuremath{\mathcal{NP}}}

% Defining maths symbol shortcuts
\newcommand{\N}{\mathbb{N}}
\newcommand{\Z}{\mathbb{Z}}
\newcommand{\R}{\mathbb{R}}
\newcommand{\F}{\mathbb{F}}
\newcommand{\C}{\mathbb{C}}
\newcommand{\Q}{\mathbb{Q}}
\newcommand{\e}{\varepsilon}
\newcommand{\p}{\varphi}
\newcommand{\bfb}{\mathbf{b}}
\NewDocumentCommand{\ddx}{s m}{
	\IfBooleanTF {#1}
		{ \frac{d}{dx}#2 }
		{ \frac{d}{dx}\left(#2\right) }
}
\newcommand{\imp}{\Longrightarrow}
\newcommand{\stline}{\;\middle\vert\;}
\newcommand{\floor}[1]{\left\lfloor#1\right\rfloor}
\newcommand{\ceil}[1]{\left\lceil#1\right\rceil}
\newcommand{\inv}{^{-1}}
\newcommand{\norm}[1]{\left\lVert#1\right\rVert}
\newcommand{\abs}[1]{\left\lvert#1\right\rvert}
\newcommand{\lrangle}[1]{\left\langle#1\right\rangle} % Inner product
\newcommand{\sub}{\leqslant}
\newcommand{\nsub}{\triangleleft}
\newcommand{\card}[1]{\overline{\overline{#1}}}

% Defining some maths operators
\DeclareMathOperator{\tr}{tr}
\DeclareMathOperator{\lcm}{lcm} % gcd already exists
\DeclareMathOperator{\proj}{proj}
\DeclareMathOperator{\sgn}{sgn}
\DeclareMathOperator{\Span}{Span\,}
\DeclareMathOperator{\Nul}{Nul}
\DeclareMathOperator{\Col}{Col}
\DeclareMathOperator{\Row}{Row}
\DeclareMathOperator{\Area}{Area}
\DeclareMathOperator{\Arg}{Arg}
\DeclareMathOperator{\Img}{Im} % \Im is already something
\DeclareMathOperator{\Ker}{Ker}
\DeclareMathOperator{\Sym}{Sym}
\DeclareMathOperator{\Stab}{Stab}
\DeclareMathOperator{\Null}{Null}
\DeclareMathOperator{\Range}{Range}
\DeclareMathOperator{\Ann}{Ann}

% Define a question environment
\NewDocumentEnvironment{question}{m o}
{\IfNoValueTF{#2}{\section*{Question #1}}{\section*{Question #1: #2}}}
{
	\clearpage\newpage
	\setcounter{thm}{0}
	\setcounter{lem}{0}
	\setcounter{prop}{0}
	\setcounter{defn}{0}
	\setcounter{case}{0}
	\setcounter{equation}{0}
}


\usepackage{amssymb, amsmath, amsthm} % https://tex.stackexchange.com/a/32102
\usepackage{mathtools} % makes maths nicer
\usepackage[margin=1in]{geometry} % Gives smaller margins
\usepackage[colorlinks=true,linkcolor=blue,citecolor=magenta]{hyperref} % For urls and to make references nicer

% http://www.ams.org/arc/tex/amscls/amsthdoc.pdf for theorems

% Possible packages:
% \usepackage{biblatex} % For bibliographies
\usepackage{graphicx} % For including images
\usepackage{float} % For [H] option for figures
\usepackage{enumitem} % For controlling lists better: see environments defiend below
\usepackage{xparse} % For helping with arguments of new environments/commands
\usepackage{bm} % For better bold maths (use \bm)
\usepackage{caption} % Makes sure that empty figure captions don't have a ":"
\usepackage{wrapfig} % to make wrapfigures - pretty self-explanatory
\usepackage{pdfpages} % To include other pdf pages (e.g. coversheets), use \includepdf[pages=-]{file.pdf} to include all of file.pdf .
\usepackage{fancyvrb} % To do fancy stuff with verbatim environments
\usepackage{polynom} % For poly long div with \polylongdiv

% Cleveref package for nice references
\usepackage[capitalise,noabbrev]{cleveref} % capitalise figure names and don't abbreviate
\crefname{equation}{}{} % Remove 'equation' from equation references

% Tikz library for drawings
\usepackage{tikz}
\usetikzlibrary{arrows.meta} % For drawing arrows in tikz
\usetikzlibrary{calc} % For calculating coordinates in tikz

\usepackage{lmodern} % Different monospace font
\usepackage[T1]{fontenc} % idk why
\usepackage{listings} % For including code
\usepackage{lstautogobble} % Remove leading tabs
\lstset{basicstyle=\ttfamily\bfseries, columns=fixed, tabsize=4, autogobble=true}
% You might want to add an autocomplete to your editor for:
% \begin{center}\vspace{-1.5em}\begin{tabular}{c}\begin{lstlisting}
% \end{lstlisting}\end{tabular}\vspace{-1em}\end{center}
% That will make a centered code listing.

% Change the document-wide spacings
\setlength{\parindent}{0in} % Make sure paragraphs don't indent
\setlength{\parskip}{1em} % Put a space between paragraphs
% Ensure that display maths environments aren't too spaced out
\setlength{\abovedisplayskip}{0.8em}
\setlength{\abovedisplayshortskip}{0.5em}
\setlength{\belowdisplayskip}{0.8em}
\setlength{\belowdisplayshortskip}{0.5em}

\allowdisplaybreaks % Lets equations run over pages

% amsthm definitions
\newtheoremstyle{plain}
	{2\topsep}  % ABOVESPACE
	{0}   		% BELOWSPACE
	{\itshape}  % BODYFONT
	{0pt}       % INDENT (empty value is the same as 0pt)
	{\bfseries} % HEADFONT
	{.}         % HEADPUNCT
	{5pt plus 1pt minus 1pt} % HEADSPACE
	{}          % CUSTOM-HEAD-SPEC

\newtheoremstyle{case}
	{6pt}
	{6pt}
	{}
	{}
	{\bfseries}
	{:}
	{.5em}
	{}

\theoremstyle{plain}
\newtheorem{thm}{Theorem}
\newtheorem{lem}{Lemma}
\newtheorem{prop}{Proposition}

\theoremstyle{definition}
\newtheorem{defn}{Definition}

\theoremstyle{case}
\newtheorem{case}{Case}
\newtheorem*{BC}{Base Case}
\newtheorem*{BCs}{Base Cases}
\newtheorem*{IS}{Inductive Step}
\newtheorem*{IF}{Proof of ``if''}
\newtheorem*{OI}{Proof of ``only if''}

% For making an enumerate environment with the list items set into the margins of the page.
\newenvironment{insetenum}[1][]
{\begin{enumerate}[leftmargin=0pt,#1]}
{\end{enumerate}}

% For indenting things
\newenvironment{indentblock}
{\begin{itemize}[parsep=\parskip,topsep=0pt,partopsep=0pt,leftmargin=0.5cm]\item[]}
{\end{itemize}}

% Defines a compact enumerate and itemize with an optional argument to specify the top-spacing (default is -1em to cancel the \parskip).
\newenvironment{compenum}[1][-0.5em]
{\begin{enumerate}[noitemsep,topsep=#1,parsep=0.5ex,partopsep=0pt]}
{\end{enumerate}}

\newenvironment{compitem}[1][-0.5em]
{\begin{itemize}[noitemsep,topsep=#1,parsep=0pt,partopsep=0pt]}
{\end{itemize}}

% Use this when there are multiple steps in a proof of something, like each
% side of an inclusion or proving an isomorphism by proving homomorphism,
% surjective, injective.
\newenvironment{pflist}[1][-0.5em]
{\begin{itemize}[topsep=#1,partopsep=0pt]}
{\end{itemize}}

% https://tex.stackexchange.com/a/35934/243466
% I don't really use this much
\usepackage{changepage}
\newenvironment{ind}[1][2em]{\begin{adjustwidth}{#1}{}}{\end{adjustwidth}}

% Complexity classes
\renewcommand{\P}[0]{\ensuremath{\mathcal{P}}}
\newcommand{\NP}[0]{\ensuremath{\mathcal{NP}}}

% Defining maths symbol shortcuts
\newcommand{\N}{\mathbb{N}}
\newcommand{\Z}{\mathbb{Z}}
\newcommand{\R}{\mathbb{R}}
\newcommand{\F}{\mathbb{F}}
\newcommand{\C}{\mathbb{C}}
\newcommand{\Q}{\mathbb{Q}}
\newcommand{\e}{\varepsilon}
\newcommand{\p}{\varphi}
\newcommand{\bfb}{\mathbf{b}}
\NewDocumentCommand{\ddx}{s m}{
	\IfBooleanTF {#1}
		{ \frac{d}{dx}#2 }
		{ \frac{d}{dx}\left(#2\right) }
}
\newcommand{\imp}{\Longrightarrow}
\newcommand{\stline}{\;\middle\vert\;}
\newcommand{\floor}[1]{\left\lfloor#1\right\rfloor}
\newcommand{\ceil}[1]{\left\lceil#1\right\rceil}
\newcommand{\inv}{^{-1}}
\newcommand{\norm}[1]{\left\lVert#1\right\rVert}
\newcommand{\abs}[1]{\left\lvert#1\right\rvert}
\newcommand{\lrangle}[1]{\left\langle#1\right\rangle} % Inner product
\newcommand{\sub}{\leqslant}
\newcommand{\nsub}{\triangleleft}
\newcommand{\card}[1]{\overline{\overline{#1}}}

% Defining some maths operators
\DeclareMathOperator{\tr}{tr}
\DeclareMathOperator{\lcm}{lcm} % gcd already exists
\DeclareMathOperator{\proj}{proj}
\DeclareMathOperator{\sgn}{sgn}
\DeclareMathOperator{\Span}{Span\,}
\DeclareMathOperator{\Nul}{Nul}
\DeclareMathOperator{\Col}{Col}
\DeclareMathOperator{\Row}{Row}
\DeclareMathOperator{\Area}{Area}
\DeclareMathOperator{\Arg}{Arg}
\DeclareMathOperator{\Img}{Im} % \Im is already something
\DeclareMathOperator{\Ker}{Ker}
\DeclareMathOperator{\Sym}{Sym}
\DeclareMathOperator{\Stab}{Stab}
\DeclareMathOperator{\Null}{Null}
\DeclareMathOperator{\Range}{Range}
\DeclareMathOperator{\Ann}{Ann}

% Define a question environment
\NewDocumentEnvironment{question}{m o}
{\IfNoValueTF{#2}{\section*{Question #1}}{\section*{Question #1: #2}}}
{
	\clearpage\newpage
	\setcounter{thm}{0}
	\setcounter{lem}{0}
	\setcounter{prop}{0}
	\setcounter{defn}{0}
	\setcounter{case}{0}
	\setcounter{equation}{0}
}
